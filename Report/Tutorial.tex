\documentclass[12pt, a4paper]{article}
\usepackage[british]{babel}
\usepackage[utf8]{inputenc}
\usepackage{amsmath}
\usepackage{amssymb}
\usepackage{amsthm}
\usepackage{booktabs}
\usepackage[hidelinks]{hyperref}
\usepackage{xcolor}
\usepackage{listings}
\definecolor{mygray}{rgb}{0.97,0.97,0.97}
\definecolor{mygreen}{rgb}{0, 0.6, 0}
\lstset{
	backgroundcolor=\color{mygray},
	language=C++,
	%    basicstyle=\footnotesize\ttfamily,
	basicstyle=\fontsize{9}{10.5}\selectfont\ttfamily,
	tabsize=3,
	numbers=left,
	numberstyle=\tiny,
	numbersep=5pt,
	stepnumber=5,
	breaklines=true,
	showstringspaces=false,
	keywordstyle=\color{blue},
	commentstyle=\color{mygreen},
	stringstyle=\color{red},
	keepspaces=true,
	columns=fullflexible
}

\newcommand{\code}[1]{{\footnotesize\ttfamily #1}}

\theoremstyle{definition}
\newtheorem*{remark}{Remark}

\title{\textbf{Introduction to the library PolyDG}}

\begin{document}
\maketitle

PolyDG is a library for the approximation of elliptic problems with 
Discontinuous Galerkin finite element methods on polyhedral grids.
It makes use of the technique of \emph{expression templates} in order to speed 
up the assembling of the matrix and of the right-hand-side of the linear 
system, moreover this approach allows a flexible implementation of different 
models starting from the variational formulation of the problem.\\
It is written in C++11.

\section{DG finite element methods}
Discontinuous Galerkin methods have shown to be very flexible and have been
successfully applied to hyperbolic, elliptic and parabolic problems arising 
from many different fields of application. Moreover one of the main advantages 
with respect to the continuous framework is the possibility of handling meshes 
with hanging nodes and made of general-shaped elements without any difficulty.\\
Meshes made of general polyhedral elements can be useful in many problems, 
especially when	we have to deal with domains that present small details or 
microstructures; these features would need too many  ``classical" 
tetrahedral/hexahedral elements to be described and so too many degrees of 
freedom.

\section{Getting started}
If you want to write a program that solves an elliptic partial differential
equation, you have to deal with three subsequent steps: the reding of the 
\code{Mesh} from a file, the creation of the \code{FeSpace} and the 
instantation of the \code{Problem}.

\begin{remark}
	You have to use the namespace \code{PolyDG} to access each class, function 
	and	type in the library (there are only few functions in the namespace 
	Utilities). For brevity we will neglect it in the following examples.
\end{remark}

\subsection{Reading the mesh}
\begin{lstlisting}
// Define a MeshReader
MeshReaderPoly reader;
// Define a Mesh
Mesh Th("fileName.mesh", reader);
// Print some information
Th.printInfo();
\end{lstlisting}

The reading of the mesh is simple, you have to declare a \code{MeshReader} and
to pass it to the constructor of \code{Mesh} together with the name of the file
containing the mesh.\\
The library provides a class \code{MeshReaderPoly} that inherits from 
\code{MeshReader} and can be used to read tridimensional tetrahedral MEDIT text 
meshes with the extension *.mesh (see 
\url{https://www.ljll.math.upmc.fr/frey/publications/RT-0253.pdf} for more 
information) or a polyhedral meshes generated with METIS (see 
\url{http://glaros.dtc.umn.edu/gkhome/metis/metis/overview}) using the bash
script provided with the meshes used for the examples.\\
If you need to read a mesh stored in another format, you can write your own
reader \emph{myReader} inheriting from the class \code{MeshReader}. The class 
has to implement a method \code{read(Mesh\& mesh, const std::string\& 
fileName)} that performs the actual reading from the file \code{fileName} and 
fills the containers of \code{Th} using a \code{MeshProxy}. In particular you 
have to read vertices with their coordinates, tetrahdra with their vertices and 
polyhedra in which are contained, external faces with their label and polyhedra 
with the tetrahedra that contain.\\
\subsection{Creating the FeSpace}
\begin{lstlisting}
// Degree of exactness for the quadrature rule over tetrahedra
unsigned quad3dDoe = 2;
// Degree of exactness for the quadrature rule over triangles
unsigned quad2dDoe = 4;
// Degree of the polynomials in the FeSpace
unsigned degree    = 2;
// Define a FeSpace
FeSpace Vh(Th, degree, quad3dDoe, quad2dDoe);
// Print some information
Vh.printInfo();
\end{lstlisting}

The creation of the FeSpace is simple too, in the constructor you have to pass	
the Mesh @c Th and the degree of the polynomials in the FeSpace.\\

\begin{remark}
The method makes use of legendre polynomials to construct the basis of the
FeSpace through the functions \code{legendre(unsigned n, Real x)} and 
\code{legendreDer(unsigned n, Real x)},	they are implemented up to \code{n = 
8}.
\end{remark}

If you want you can choose the quadrature rules you want to use specifying the
degrees of exactness. Quadrature rules are stores in \code{QuadRuleManager}. 
Actually there are rules over tetrahedra with degree of exactness from 1 to 8 
and quadrature rules over triangles with degree of exactness from 1 to 10. Some 
of them have one negative weight.\\
You can add a new rule with \code{QuadRuleManager::setTetraRule(const 
QuadRule3D\& rule)} and \code{QuadRuleManager::setTriaRule(const QuadRule2D\& 
rule)}.

\subsection{Instantation of the Problem}
\begin{lstlisting}
// Define a problem
Problem poisson(Vh);
\end{lstlisting}

The instantation of the Problem is the key part of the program.
To the constructor you have to pass only the FeSpace that is used for the
approximation of the solution and for the test functions.\\
Then there are three substeps: the definition of the variational formulation,
the integration and the solution of the linear system. Let us suppose we
want to solve the Poisson problem with Dirchlet boundary conditions, then the 
variational formulation is:\\
find $ u \in H^s(\mathcal{T}), s>3/2 $, such that:
\begin{multline*}
			\sum_{\kappa \in \mathcal{T}} \int_\kappa \nabla u \cdot \nabla v
			-\sum_{e \in \Gamma} \bigg( \int_e [v] \cdot \{\!\!\{ \nabla u \}\!\!\}
			+ \int_e [u] \cdot \{\!\!\{ \nabla v \}\!\!\}
			- \gamma_e \int_e [u] \cdot [v] \bigg)\\
			= \sum_{\kappa \in \mathcal{T}} \int_\kappa fv
			+ \sum_{e \in \Gamma_D} \bigg( - \int_e g \nabla v \cdot \mathbf{n}
			+ \gamma_e \int_e gv \bigg) \quad \forall v \in H^s(\mathcal{T})
\end{multline*}
where $ \mathcal{T} $ is the mesh, $ \Gamma_D $ is the set of
external faces with Dirichlet boundary conditions and $ \Gamma = \Gamma_h
\cup \Gamma_D $, with $ \Gamma_h $ the set of internal faces.\\

\subsubsection{Operators}
\begin{lstlisting}
// Test function
PhiI            v;
// Gradient of the solution
GradPhiJ        uGrad;
// Gradient of the test function
GradPhiI        vGrad;
// Jump of the solution
JumpPhiJ        uJump;
// Jump of the test function
JumpPhiI        vJump;
// Average of the gradient of the solution
AverGradPhiJ    uGradAver;
// Average of the gradient of the test function
AverGradPhiI    vGradAver;
// Penalty scaling for the penalization term
PenaltyScaling  gamma(10.0);
// Normal vector to faces
Normal          n;
// Source function
Function        f(source);
// Dirichlet datum
Function        gd(uex);
\end{lstlisting}
First you have to declare the operators you want to use to compose the
variational formulation. Basically they are callable objects and once defined
you can use them many times.\\
\begin{remark}
Operators that end with \code{J} are related to the solution, operators
that end with \code{I} are related to the test function.
\end{remark}

\subsubsection{Integration}
\begin{lstlisting}
// Integration of the bilinear form over the volume of the elements
poisson.integrateVol(dot(uGrad, vGrad), true);
// Integration of the bilinear form over the external faces
poisson.integrateFacesExt(-dot(uGradAver, vJump) - dot(uJump, vGradAver) + 
gamma * dot(uJump, vJump), {1, 2, 3, 4, 5, 6}, true);
// Integration of the bilinear form over the internal faces
poisson.integrateFacesInt(-dot(uGradAver, vJump) - dot(uJump, vGradAver) + 
gamma * dot(uJump, vJump), true);
// Integration of the rhs over the volume of the elements
poisson.integrateVolRhs(f * v);
// Integration of the rhs over the external faces
poisson.integrateFacesExtRhs(-gd * dot(n, vGrad) + gamma * gd * v, {1, 2, 3, 4, 
5, 6});

// Assembly of the matrix
poisson.finalizeMatrix();
\end{lstlisting}

Then you have to insert the variational formulation that may consist of
integrals over volumes, external faces and integrnal faces. You can une
the operator dot to perform scalar products, while you can use \code{+,*,-,/}
for algebraic operations with real numbers.\\
In the integrals over external faces you have to specify over which faces
you want to integrate, using the labels. In the example above 
\code{{1,2,3,4,5,6}} stands for a \code{std::vector<BCLabelType>} containing 
the label of faces,	that in this case are all the faces of the cube.\\
For each of the integrals of the bilinear form, you can specify if it is
symmetric or not with respect to the solution and the test function (if
not specified it is assumed to be non-symmetric). If an integral is symmetric
it can be computed for half of the basis function, with a noticible reduction
in the computational cost. Moreover if the bilinear form is symmetric so that
all the integrals are symmetric then a symmetric matrix is stored, with another
noticible memory saving.\\
Finally remember to call the method \code{Problem::finalizeMatrix()} that is 
needed to actually assemble the matrix.\\

\subsubsection{Solution of the linear system}
\begin{lstlisting}
// Solve with LU sparse decomposition
poisson.solveLU();
// Solve with Chlolesky sparse decomposition
poisson.solveCholesky();
// Solve with conjugate gradient
poisson.solveCG(Eigen::VectorXd::Zero(poisson.getDim()), 2 * poisson.getDim(), 
1e-10);
// Solve with BiCGSTAB
poisson.solveBiCGSTAB(Eigen::VectorXd::Zero(poisson.getDim()), 2 * 
poisson.getDim(), 1e-10);
\end{lstlisting}

Eventually the solution of the linear system can be obtained through four
methods provided by the class Problem exploiting the library Eigen
(see \url{https://eigen.tuxfamily.org/dox/group__TopicSparseSystems.html} for 
more information).\\
The iterative ones need an initial guess and allow to set a maximum number
of iteration and a tolerance for the convergence.

\begin{lstlisting}
// Export the solution
poisson.exportSolutionVTK("solution.vtu");
// Compute the L2 norm of the error
Real errL2  = computeErrorL2(uex);
// Compute the H1 seminorm of the error
Real errH10 = computeErrorH10(uexGrad);
\end{lstlisting}

At very last you can export the solution in a VTK format in order to visualize 
it or you can compute the $L^2$ norm and $H^1$ seminorm of the error if you 
know the 
analytical solution and its gradient.

\end{document}